%------------------------------------------------
\subsection{Random Forest}
%------------------------------------------------

\begin{frame}
\frametitle{Random Forest}
\framesubtitle{Definition}
%\begin{table}
%\begin{tabular}{l l l}
%\toprule
%\textbf{Treatments} & \textbf{Response 1} & \textbf{Response 2}\\
%\midrule
%Treatment 1 & 0.0003262 & 0.562 \\
%Treatment 2 & 0.0015681 & 0.910 \\
%Treatment 3 & 0.0009271 & 0.296 \\
%\bottomrule
%\end{tabular}
%\caption{Table caption}
%\end{table}


\vspace{1ex}


\begin{definition}[by L.Breiman \cite{breiman2001random}]
	A random forest is a classifier consisting of a collection of tree-structured classifiers ${\hat{T}_{\theta_{b}}(\textbf{x})}, b = 1,...,B$ where the $\theta_{b}$ are independent identically
	distributed random vectors and each tree casts a unit vote for the most popular class at input $\textbf{x}$.
\end{definition}
\vspace{4ex}

Random Forest is an extension and improvement over bagging:
\vspace{1ex}
\begin{enumerate}
\item Like in bagging, multiple decision trees are built
\vspace{1ex}
\item Improvement: an injection of randomness is made
\end{enumerate}

\end{frame}

%------------------------------------------------

\begin{frame}
\frametitle{Random Forest}
\framesubtitle{Randomness in the model}

Two key concepts that makes decision forest "random" are \cite{friedman2001elements}:
\vspace{1ex}
\begin{enumerate}
	\item Random sampling of training data points when building trees
\vspace{1ex}
	\item Random subsets of variables considered when splitting nodes. Recommended number of variables:
\vspace{1ex}
	\begin{enumerate}[a]
	    \item For classification:  $\lfloor{\sqrt{n}} \rfloor$
\vspace{1ex}
	    \item For regression: $\lfloor \frac{n}{3} \rfloor$
	\end{enumerate}
\end{enumerate}


\end{frame}

\begin{frame}
\frametitle{Random Forest}
\framesubtitle{Algorithm}

\begin{algorithm}[H]
\SetAlgoLined
\begin{enumerate}
	\bigbreak
	\item For $b$ = 1 to $B$:
	\begin{enumerate}[a]
	    \item Draw a bootstrap sample $D_{b}$ of size N from the training data.
	    \item Grow the Random Forest tree ${{T}_{D_{b},\theta_{b}}}$ to the bootstrapped data, by recursively repeating the following steps for each terminal node of the tree, until the minimum node size $n_{min}$ is reached:
	    \begin{enumerate}[i]
	       \item Select $m$ variables denoted by $\theta_{b}$ at random from the $n$ variables
	       \item Pick the best variable/split-point among the $m$
	       \item  Split the node into two daughter nodes
	    \end{enumerate}
	\end{enumerate}
	\item  Output the ensemble of trees $\{{T}_{D_{b},\theta_{b}}\}_{b=1}^{B}$
	\smallbreak
\end{enumerate}
 \caption{Random Forest for Regression or Classification \cite{friedman2001elements}}
\end{algorithm}

\end{frame}















