\documentclass{beamer}
\usepackage{dsfont}
\usepackage{fontawesome}
\mode<presentation> {

%\usetheme{default}
%\usetheme{AnnArbor}
%\usetheme{Antibes}
%\usetheme{Bergen}
%\usetheme{Berkeley}
%\usetheme{Berlin}
%\usetheme{Boadilla}
%\usetheme{CambridgeUS}
%\usetheme{Copenhagen}
%\usetheme{Darmstadt}
%\usetheme{Dresden}
%\usetheme{Frankfurt}
%\usetheme{Goettingen}
%\usetheme{Hannover}
%\usetheme{Ilmenau}
%\usetheme{JuanLesPins}
%\usetheme{Luebeck}
\usetheme{Madrid}
%\usetheme{Malmoe}
%\usetheme{Marburg}
%\usetheme{Montpellier}
%\usetheme{PaloAlto}
%\usetheme{Pittsburgh}
%\usetheme{Rochester}
%\usetheme{Singapore}
%\usetheme{Szeged}
%\usetheme{Warsaw}

%\usecolortheme{albatross}
%\usecolortheme{beaver}
%\usecolortheme{beetle}
%\usecolortheme{crane}
%\usecolortheme{dolphin}
%\usecolortheme{dove}
%\usecolortheme{fly}
%\usecolortheme{lily}
%\usecolortheme{orchid}
%\usecolortheme{rose}
%\usecolortheme{seagull}
%\usecolortheme{seahorse}
%\usecolortheme{whale}
%\usecolortheme{wolverine}

%\setbeamertemplate{footline} % To remove the footer line in all slides uncomment this line
%\setbeamertemplate{footline}[page number] % To replace the footer line in all slides with a simple slide count uncomment this line

%\setbeamertemplate{navigation symbols}{} % To remove the navigation symbols from the bottom of all slides uncomment this line
}

\usepackage{graphicx} % Allows including images
\usepackage{booktabs} % Allows the use of \toprule, \midrule and \bottomrule in tables

\usepackage[backend=biber, bibencoding=utf8]{biblatex}
\usepackage[ruled, vlined]{algorithm2e}
\usepackage{amssymb}
\usepackage{amsmath}
\usepackage{amsfonts}
\usepackage{multirow}
\setbeamercovered{transparent}
\newcommand{\semitransp}[2][20]{\color{fg!#1}#2}
\newcommand{\nontransp}[2][100]{\color{fg!#1}#2}
\bibliography{references.bib}



\newcommand{\irow}[1]{% inline row vector
  \begin{smallmatrix}(#1)\end{smallmatrix}%
}

%----------------------------------------------------------------------------------------
%	TITLE PAGE
%----------------------------------------------------------------------------------------

\title{Random Forest for Classification Problems} % The short title appears at the bottom of every slide, the full title is only on the title page
\author[Redmer, Modzelewski, Balaban]{Raphael Redmer, Arkadiusz Modzelewski,\newline Burak Balaban}
\institute[Uni Bonn]{University of Bonn \\ Research Module in Econometrics and Statistics}
\date{January 20, 2020} % Date, can be changed to a custom date

\begin{document}

\begin{frame}
    \titlepage % Print the title page as the first slide
\end{frame}

\begin{frame}
\frametitle{Overview} % Table of contents slide, comment this block out to remove it
\tableofcontents % Throughout your presentation, if you choose to use \section{} and \subsection{} commands, these will automatically be printed on this slide as an overview of your presentation
\end{frame}

%----------------------------------------------------------------------------------------
%	PRESENTATION SLIDES
%----------------------------------------------------------------------------------------

%\include{slides/01_intro}

\section{From Tree to Random Forest}
\subsection{Decision Tree}

\begin{frame}
	\frametitle{Decision Tree}
	\framesubtitle{Example}
	\begin{figure}		
		\includegraphics[height=0.7\textheight]{images/decision_tree_example.png}
	\end{figure}
\end{frame}

%%%% Decision Tree: Tree Building Process
\begin{frame}
	\frametitle{Decision Tree}
	\framesubtitle{Tree Building Process}
	A tree is grown starting from the root node by repeatedly 
	using the following steps on each node (also called binary splitting)\cite{breiman1984classification}:
	\smallbreak
	\begin{enumerate}
		\onslide<1->{
			\item[(i)] Find best split \(s\) for each feature \(X_{m}\)
		}
		\onslide<2->{
			\item[(ii)] Find the best split of the node
		}
		\onslide<3->{
			\item[(iii)] Repeat until stopping criterion got satisfied
		}
	\end{enumerate}
\end{frame}	

%%%% Decision Tree: Purity Measures
\begin{frame}
	\frametitle{Decision Tree}
	\framesubtitle{Purity Measures}
	\begin{block}{Gini Measure}
		\begin{equation*}    
			i(t) = \sum_{c \in C} p(c|t) (1 - p(c|t))
		\end{equation*}
	\end{block}
	\begin{block}{Information Entropy}
		\begin{equation*}    
			i(t) = \sum_{c \in C} p(c|t) log(p(c|t))
		\end{equation*}
	\end{block}	
	\bigbreak
	where $C$ is the set of classes $c$ and $t$ a node of the tree.
\end{frame}



\begin{frame}[t]
    \frametitle{Bias Variance Trade-off}
    \framesubtitle{The Expected Generalization Error}
    \onslide<1->{
            $y = f(x) + \epsilon$ and 
            $\epsilon \sim \mathcal{N}(0,\sigma_{\epsilon}^2)$\\
            \smallbreak
            Estimate of $f(x)$:$\quad \hat{f}(x)$\\
            \smallbreak
            The expected generalization error: $\boldsymbol{Err}(\hat{f}(x))$ \\
    }
    \bigbreak
    \onslide<2->{
        The decomposition of a model's expected generalization error is
        \begin{block}{}
            \begin{center}
                $\boldsymbol{Err}(\hat{f}(x)) = \sigma_{\epsilon}^2 + [Bias(\hat{f}(x))]^2 + Var(\hat{f}(x))$
            \end{center}
        \end{block}
    }
    \bigbreak
    \onslide<3->{
        \quad $\sigma_{\epsilon}^2$ is irreducible and independent of the model. \\
        \bigbreak
        \quad Trade-off between bias and variance. \\
        \smallbreak
        \quad \textbf{Aim:} Decrease variance while keeping bias unincreased.
        }
\end{frame}

\begin{frame}
    \frametitle{Bias-Variance Trade-off}
    \framesubtitle{Illustration}
    \begin{figure}      
        \includegraphics[height=0.6\textheight]{images/bias-variance.png}
    \end{figure}
    \begin{center}
    Decision trees generally have low bias and high variance.
    \end{center}
\end{frame}

    %\begin{columns}[c] % The "c" option specifies centered vertical alignment while the "t" option is used for top vertical alignment
    
    %\column{.45\textwidth} % Left column and width
    %\textbf{Heading}
    %\begin{enumerate}
    %\item Statement
    %\item Explanation
    %\item Example
    %\end{enumerate}
    
    %\column{.5\textwidth} % Right column and width
    %Lorem ipsum dolor sit amet, consectetur adipiscing elit. Integer lectus nisl, ultricies in feugiat rutrum, porttitor sit amet augue. Aliquam ut tortor mauris. Sed volutpat ante purus, quis accumsan dolor.
    
    %\end{columns}


\subsection{Bagging}

\begin{frame}
    \frametitle{Bagging}
    \framesubtitle{Improvement}
	Bagging is:
    \begin{enumerate}
    \item created for methods with high variance
    \item reduces variance and gives better predictions
    \item improvement of bagging: Random Forest
    \end{enumerate}
    

\end{frame}

\begin{frame}
    \frametitle{Bagging}
  \framesubtitle{Ilustration}

	\begin{center}		
		\includegraphics[height=0.7\textheight]{images/bagging_1.jpg}
	\end{center}

\end{frame}

%------------------------------------------------
\subsection{Random Forest}
%------------------------------------------------

\begin{frame}
\frametitle{Random Forest}
\framesubtitle{Definition}
%\begin{table}
%\begin{tabular}{l l l}
%\toprule
%\textbf{Treatments} & \textbf{Response 1} & \textbf{Response 2}\\
%\midrule
%Treatment 1 & 0.0003262 & 0.562 \\
%Treatment 2 & 0.0015681 & 0.910 \\
%Treatment 3 & 0.0009271 & 0.296 \\
%\bottomrule
%\end{tabular}
%\caption{Table caption}
%\end{table}


\vspace{1ex}


\begin{definition}[by L.Breiman \cite{breiman2001random}]
	A random forest is a classifier consisting of a collection of tree-structured classifiers ${\hat{T}_{\theta_{b}}(\textbf{x})}, b = 1,...,B$ where the $\theta_{b}$ are independent identically
	distributed random vectors and each tree casts a unit vote for the most popular class at input $\textbf{x}$.
\end{definition}
\vspace{4ex}

Random Forest is an extension and improvement over bagging:
\vspace{1ex}
\begin{enumerate}
\item Like in bagging, multiple decision trees are built
\vspace{1ex}
\item Improvement: an injection of randomness is made
\end{enumerate}

\end{frame}

%------------------------------------------------

\begin{frame}
\frametitle{Random Forest}
\framesubtitle{Randomness in the model}

Two key concepts that makes decision forest "random" are \cite{friedman2001elements}:
\vspace{1ex}
\begin{enumerate}
	\item Random sampling of training data points when building trees
\vspace{1ex}
	\item Random subsets of variables considered when splitting nodes. Recommended number of variables:
\vspace{1ex}
	\begin{enumerate}[a]
	    \item For classification:  $\lfloor{\sqrt{n}} \rfloor$
\vspace{1ex}
	    \item For regression: $\lfloor \frac{n}{3} \rfloor$
	\end{enumerate}
\end{enumerate}


\end{frame}

\begin{frame}
\frametitle{Random Forest}
\framesubtitle{Algorithm}

\begin{algorithm}[H]
\SetAlgoLined
\begin{enumerate}
	\bigbreak
	\item For $b$ = 1 to $B$:
	\begin{enumerate}[a]
	    \item Draw a bootstrap sample $D_{b}$ of size N from the training data.
	    \item Grow the Random Forest tree ${{T}_{D_{b},\theta_{b}}}$ to the bootstrapped data, by recursively repeating the following steps for each terminal node of the tree, until the minimum node size $n_{min}$ is reached:
	    \begin{enumerate}[i]
	       \item Select $m$ variables denoted by $\theta_{b}$ at random from the $n$ variables
	       \item Pick the best variable/split-point among the $m$
	       \item  Split the node into two daughter nodes
	    \end{enumerate}
	\end{enumerate}
	\item  Output the ensemble of trees $\{{T}_{D_{b},\theta_{b}}\}_{b=1}^{B}$
	\smallbreak
\end{enumerate}
 \caption{Random Forest for Regression or Classification \cite{friedman2001elements}}
\end{algorithm}

\end{frame}

















\section{Bias Variance}

\begin{frame}[fragile] % Need to use the fragile option when verbatim is used in the slide
    \frametitle{Mathematical Concept}
    \begin{example}[Theorem Slide Code]
    \begin{verbatim}
    \begin{frame}
    \frametitle{Theorem}
    \begin{theorem}[Mass--energy equivalence]
    $E = mc^2$
    \end{theorem}
    \end{frame}\end{verbatim}
    \end{example}
\end{frame}



\section{Simulation Study}
\frame{\sectionpage}
\frametitle{Simulation Study}
\begin{frame}
    \frametitle{Simulation Study}
    \framesubtitle{Linear DGP}
    The linear DGP generates the data tuples \( (y, x_{1}, x_{2}, x_{3}) \) as follows:

    \begin{equation*}\label{eq:linear_dgp}
        y = \beta_{0} + \beta_{1} x_{1} + \beta_{2} x_{2} + \beta_{3} x_{3} + \epsilon,
    \end{equation*}
    
    whereas
    \begin{itemize}
        \item $ (\beta_{0}, \beta_{1}, \beta_{2}, \beta_{3} ) = (0.3, 5, 10, 15)$
        \item $x_{1}, x_{2}, x_{3} \sim \mathcal{N}(0,\,3)$
        \item $\epsilon \sim \mathcal{N}(0,\,1)$
    \end{itemize}
\end{frame}

\begin{frame}
    \frametitle{Simulation Study}
    \framesubtitle{Linear DGP Results}
	\begin{center}		
		\includegraphics[height=0.7\textheight]{images/forest_vs_ols_linearDGP.png}
	\end{center}
\end{frame}


\begin{frame}
    \frametitle{Simulation Study}
    \framesubtitle{Non-Linear DGP}
    The non-linear DGP generates the data tuples \( (y, x_{1}, x_{2}) \) as follows:

    \begin{equation*}\label{eq:non_linear_dgp}
        y = \beta_{0} + \beta_{1} \mathds{1}(x_{1} \geq 0, x_{2} \geq 0) + \beta_{2} \mathds{1}(x_{1} \geq 0, x_{2} < 0) + \beta_{3} \mathds{1}(x_{1} < 0) + \epsilon,
    \end{equation*}
    
    whereas
    \begin{itemize}
        \item $ (\beta_{0}, \beta_{1}, \beta_{2}, \beta_{3} ) = (0.3, 5, 10, 15)$
        \item $x_{1}, x_{2} \sim \mathcal{N}(0,\,3)$
        \item $\epsilon \sim \mathcal{N}(0,\,1)$
    \end{itemize}
    are the same as in the previous DGP.
\end{frame}

\begin{frame}
    \frametitle{Simulation Study}
    \framesubtitle{Non-Linear DGP Results}
	\begin{center}		
		\includegraphics[height=0.7\textheight]{images/forest_vs_ols_nonlinearDGP.png}
	\end{center}
\end{frame}

\section{Real Data}

\begin{frame}[fragile] % Need to use the fragile option when verbatim is used in the slide
    \frametitle{Real Data}
    \textbf{Data}: Titanic data
\vspace{1ex}
    \newline \textbf{Method used}: 
\vspace{1ex}
    \begin{enumerate}
        \item Random Forest
        \item AdaBoost
        \item Gradient Boosting Classifier
    \end{enumerate}
\vspace{1ex}
    \textbf{Goal}: Given features of passengers predict which passengers survived the Titanic shipwreck
\end{frame}
    
%------------------------------------------------

\begin{frame}[fragile] % Need to use the fragile option when verbatim is used in the slide
    \frametitle{Real Data: results}
 
\begin{columns}[c] % The "c" option specifies centered vertical alignment while the "t" option is used for top vertical alignment
    \column{.33\textwidth} % Left column and width
    Random Forest
\vspace{1ex}
    \newline Accuracy: ~84,32\%
\vspace{2ex}

    
\begin{table}
\centering
\begin{tabular}{lll}
  & $D_{pred}$ & $S_{pred}$    \\
$D_{real}$ & 88\%   &  12\%\\
$S_{real}$ & 21\%& 79\%   \\

\end{tabular}
\end{table}


    \column{.33\textwidth} % Central column and width
    AdaBoost
\vspace{1ex}
    \newline Accuracy: ~82.8\%
\vspace{2ex}

\begin{table}
\centering
\begin{tabular}{lll}
  & $D_{pred}$ & $S_{pred}$    \\
$D_{real}$ & 86\%   &   14\%\\
$S_{real}$ &  21\% & 79\%   \\

\end{tabular}
\end{table}

	
    \column{.33\textwidth} % Right column and width
    Gradient Boosting
\vspace{1ex}
    \newline Accuracy: ~82,8\%
\vspace{2ex}

\begin{table}
\centering
\begin{tabular}{lll}
  & $D_{pred}$ & $S_{pred}$    \\
$D_{real}$ & 85\%   &  15\%\\
$S_{real}$ &  20\% &    80\%\\

\end{tabular}
\end{table}


    \end{columns}
\end{frame}

\begin{frame}
    \frametitle{Contact Data}
    \begin{itemize}
        \item Burak Balaban\\
        \qquad \faGithub /burakbalaban/\\
        \qquad burak.balaban@uni-bonn.de
        \smallbreak
        \item Raphael Redmer\\
        \qquad \faGithub /RaRedmer/\\
        \qquad ra.redmer@outlook.com
        \smallbreak
        \item Arkadiusz Modzelewski\\
        \qquad \faGithub /ArcadiusM/\\
        \qquad arcadius.modzelewski@gmail.com
    \end{itemize}
    \bigbreak
    \bigbreak
    \quad The link to presentation's repository:\\
    \vspace{-1.5cm}
    \center{
    %for the alignment
    \qquad\qquad\qquad\qquad\qquad
    \includegraphics[height=0.23\textheight]{images/qrcode.png}}
\end{frame}
    
%---------------------------------------------------------------------------------------

\section*{References}

\begin{frame}
    \frametitle{References}
    \footnotesize{
    \printbibliography
    }
\end{frame}

\end{document} 